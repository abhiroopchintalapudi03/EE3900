\documentclass[journal,12pt,twocolumn]{IEEEtran}

\usepackage{setspace}
\usepackage{gensymb}
\singlespacing
\usepackage[cmex10]{amsmath}

\usepackage{amsthm}
\usepackage[version=4]{mhchem}

\usepackage{mathrsfs}
\usepackage{txfonts}
\usepackage{stfloats}
\usepackage{bm}
\usepackage{cite}
\usepackage{cases}
\usepackage{subfig}

\usepackage{longtable}
\usepackage{multirow}

\usepackage{enumitem}
\usepackage{mathtools}
\usepackage{steinmetz}
\usepackage{tikz}
\usepackage{circuitikz}
\usepackage{verbatim}
\usepackage{tfrupee}
\usepackage[breaklinks=true]{hyperref}
\usepackage{graphicx}
\usepackage{tkz-euclide}

\usetikzlibrary{calc,math}
\usepackage{listings}
    \usepackage{color}                                            %%
    \usepackage{array}                                            %%
    \usepackage{longtable}                                        %%
    \usepackage{calc}                                             %%
    \usepackage{multirow}                                         %%
    \usepackage{hhline}                                           %%
    \usepackage{ifthen}                                           %%
    \usepackage{lscape}     
\usepackage{multicol}
\usepackage{chngcntr}

\DeclareMathOperator*{\Res}{Res}

\renewcommand\thesection{\arabic{section}}
\renewcommand\thesubsection{\thesection.\arabic{subsection}}
\renewcommand\thesubsubsection{\thesubsection.\arabic{subsubsection}}

\renewcommand\thesectiondis{\arabic{section}}
\renewcommand\thesubsectiondis{\thesectiondis.\arabic{subsection}}
\renewcommand\thesubsubsectiondis{\thesubsectiondis.\arabic{subsubsection}}


\hyphenation{op-tical net-works semi-conduc-tor}
\def\inputGnumericTable{}                                 %%

\lstset{
%language=C,
frame=single, 
breaklines=true,
columns=fullflexible
}
\begin{document}


\newtheorem{theorem}{Theorem}[section]
\newtheorem{problem}{Problem}
\newtheorem{proposition}{Proposition}[section]
\newtheorem{lemma}{Lemma}[section]
\newtheorem{corollary}[theorem]{Corollary}
\newtheorem{example}{Example}[section]
\newtheorem{definition}[problem]{Definition}

\newcommand{\BEQA}{\begin{eqnarray}}
\newcommand{\EEQA}{\end{eqnarray}}
\newcommand{\define}{\stackrel{\triangle}{=}}
\bibliographystyle{IEEEtran}
\raggedbottom
\setlength{\parindent}{0pt}
\providecommand{\mbf}{\mathbf}
\providecommand{\pr}[1]{\ensuremath{\Pr\left(#1\right)}}
\providecommand{\qfunc}[1]{\ensuremath{Q\left(#1\right)}}
\providecommand{\sbrak}[1]{\ensuremath{{}\left[#1\right]}}
\providecommand{\lsbrak}[1]{\ensuremath{{}\left[#1\right.}}
\providecommand{\rsbrak}[1]{\ensuremath{{}\left.#1\right]}}
\providecommand{\brak}[1]{\ensuremath{\left(#1\right)}}
\providecommand{\lbrak}[1]{\ensuremath{\left(#1\right.}}
\providecommand{\rbrak}[1]{\ensuremath{\left.#1\right)}}
\providecommand{\cbrak}[1]{\ensuremath{\left\{#1\right\}}}
\providecommand{\lcbrak}[1]{\ensuremath{\left\{#1\right.}}
\providecommand{\rcbrak}[1]{\ensuremath{\left.#1\right\}}}
\theoremstyle{remark}
\newtheorem{rem}{Remark}
\newcommand{\sgn}{\mathop{\mathrm{sgn}}}
\providecommand{\abs}[1]{\left\vert#1\right\vert}
\providecommand{\res}[1]{\Res\displaylimits_{#1}} 
\providecommand{\norm}[1]{\left\lVert#1\right\rVert}
%\providecommand{\norm}[1]{\lVert#1\rVert}
\providecommand{\mtx}[1]{\mathbf{#1}}
\providecommand{\mean}[1]{E\left[ #1 \right]}
\providecommand{\fourier}{\overset{\mathcal{F}}{ \rightleftharpoons}}
%\providecommand{\hilbert}{\overset{\mathcal{H}}{ \rightleftharpoons}}
\providecommand{\system}{\overset{\mathcal{H}}{ \longleftrightarrow}}
	%\newcommand{\solution}[2]{\textbf{Solution:}{#1}}
\newcommand{\solution}{\noindent \textbf{Solution: }}
\newcommand{\cosec}{\,\text{cosec}\,}
\providecommand{\dec}[2]{\ensuremath{\overset{#1}{\underset{#2}{\gtrless}}}}
\newcommand{\myvec}[1]{\ensuremath{\begin{pmatrix}#1\end{pmatrix}}}
\newcommand{\mydet}[1]{\ensuremath{\begin{vmatrix}#1\end{vmatrix}}}
\numberwithin{equation}{subsection}
\makeatletter
\@addtoreset{figure}{problem}
\makeatother
\let\StandardTheFigure\thefigure
\let\vec\mathbf
\renewcommand{\thefigure}{\theproblem}
\def\putbox#1#2#3{\makebox[0in][l]{\makebox[#1][l]{}\raisebox{\baselineskip}[0in][0in]{\raisebox{#2}[0in][0in]{#3}}}}
     \def\rightbox#1{\makebox[0in][r]{#1}}
     \def\centbox#1{\makebox[0in]{#1}}
     \def\topbox#1{\raisebox{-\baselineskip}[0in][0in]{#1}}
     \def\midbox#1{\raisebox{-0.5\baselineskip}[0in][0in]{#1}}
\vspace{3cm}
\title{QUIZ-2}
\author{Abhiroop Chintalapudi-AI20BTECH11005}
\maketitle
\newpage
\bigskip
\renewcommand{\thefigure}{\theenumi}
\renewcommand{\thetable}{\theenumi}
Download latex-tikz code from  
\begin{lstlisting}
https://github.com/abhiroopchintalapudi03/EE3900/tree/main/QUIZ-2
\end{lstlisting}

\section{Problem 3.9(c)}
A casual LTI system has an impulse response$h[n]$, for which $z$-transform is\\
\begin{equation}
H(z)=\frac{1+z^{-1}}{(1- \frac{1}{2}z^{-1})(1+\frac{1}{4}z^{-1})} \label{eq3.9.1}
\end{equation}
(c) Find the $z$-transform $X(z)$ of an input $x[n]$ that will produce the output\\
\begin{equation}
y[n]=-\frac{1}{3}(-\frac{1}{4})^{n}u[n] - \frac{4}{3}(2)^{n}u[-n-1]  \label{eq3.9.2}
\end{equation}
\section{Solution}
From $z$-transform of basic signals\\
\begin{align}
\ce{u[n] <=>[Z] \frac{z}{z-1}}\\
\end{align}

\begin{equation}
\ce{(-\frac{1}{4})^{n}u[n] <=>[Z] \frac{z}{z-(-\frac{1}{4})}}\\
\end{equation}
\begin{equation}
\ce{-\frac{1}{3}(-\frac{1}{4})^{n}u[n] <=>[Z] -\frac{1}{3}(\frac{z}{z-(-\frac{1}{4})})}\\
\end{equation}
\begin{equation}
\ce{-\frac{1}{3}(-\frac{1}{4})^{n}u[n] <=>[Z] -\frac{1}{3}(\frac{4z}{4z + 1})}\\
\end{equation}
\begin{equation}
\ce{u[-n-1] <=>[Z] -\frac{z}{z-1}}\\
\end{equation}
\begin{equation}
\ce{-\frac{4}{3}(2)^{n}u[-n-1] <=>[Z] \frac{4}{3}(\frac{z}{z-2})}\\
\end{equation}
\begin{equation}
\ce{-\frac{1}{3}(-\frac{1}{4})^{n}u[n] - \frac{4}{3}(2)^{n}u[-n-1] <=>[Z] -\frac{1}{3}(\frac{4z}{4z + 1}) + \frac{4}{3}(\frac{z}{z-2})}
\end{equation}
\newpage
\begin{equation}
\Rightarrow Y(z)= \frac{4z(z+1)}{(z-2)(4z+1)} \label{eq3.9.3}
\end{equation}
 From \eqref{eq3.9.1} we also know that,\\
\begin{equation}
H(z)=\frac{1+z^{-1}}{(1- \frac{1}{2}z^{-1})(1+\frac{1}{4}z^{-1})} = \frac{8z(z+1)}{(2Z-1)(4z+1)}
\end{equation}
From transfer function we know that,\\
\begin{equation}
H(z)=\frac{Y(z)}{X(z)}\\
\Rightarrow X(z) = \frac{Y(z)}{H(z)}
\end{equation}
$\Rightarrow$ $X(z)= \frac{\frac{4z(z+1)}{(z-2)(4z+1)}}{\frac{8z(z+1)}{(2Z-1)(4z+1)}}$\\
$\Rightarrow$ $X(z) = \frac{2z-1}{2(z-2)}=\frac{1-\frac{1}{2}z^{-1}}{1-2z^{-1}}$ \label{eq3.9.4}\\

$\Rightarrow$ The $z$-transform $X(z)$ of an input $x[n]$ that will produce the given output is\\
\begin{equation}
X(z) = \frac{1-\frac{1}{2}z^{-1}}{1-2z^{-1}} \label{eq3.9.5}
\end{equation}
\end{document}