\documentclass[journal,12pt,twocolumn]{IEEEtran}

\usepackage{setspace}
\usepackage{gensymb}
\singlespacing
\usepackage[cmex10]{amsmath}

\usepackage{amsthm}

\usepackage{mathrsfs}
\usepackage{txfonts}
\usepackage{stfloats}
\usepackage{bm}
\usepackage{cite}
\usepackage{cases}
\usepackage{subfig}

\usepackage{longtable}
\usepackage{multirow}

\usepackage{enumitem}
\usepackage{mathtools}
\usepackage{steinmetz}
\usepackage{tikz}
\usepackage{circuitikz}
\usepackage{verbatim}
\usepackage{tfrupee}
\usepackage[breaklinks=true]{hyperref}
\usepackage{graphicx}
\usepackage{tkz-euclide}

\usetikzlibrary{calc,math}
\usepackage{listings}
    \usepackage{color}                                            %%
    \usepackage{array}                                            %%
    \usepackage{longtable}                                        %%
    \usepackage{calc}                                             %%
    \usepackage{multirow}                                         %%
    \usepackage{hhline}                                           %%
    \usepackage{ifthen}                                           %%
    \usepackage{lscape}     
\usepackage{multicol}
\usepackage{chngcntr}

\DeclareMathOperator*{\Res}{Res}

\renewcommand\thesection{\arabic{section}}
\renewcommand\thesubsection{\thesection.\arabic{subsection}}
\renewcommand\thesubsubsection{\thesubsection.\arabic{subsubsection}}

\renewcommand\thesectiondis{\arabic{section}}
\renewcommand\thesubsectiondis{\thesectiondis.\arabic{subsection}}
\renewcommand\thesubsubsectiondis{\thesubsectiondis.\arabic{subsubsection}}


\hyphenation{op-tical net-works semi-conduc-tor}
\def\inputGnumericTable{}                                 %%

\lstset{
%language=C,
frame=single, 
breaklines=true,
columns=fullflexible
}
\begin{document}


\newtheorem{theorem}{Theorem}[section]
\newtheorem{problem}{Problem}
\newtheorem{proposition}{Proposition}[section]
\newtheorem{lemma}{Lemma}[section]
\newtheorem{corollary}[theorem]{Corollary}
\newtheorem{example}{Example}[section]
\newtheorem{definition}[problem]{Definition}

\newcommand{\BEQA}{\begin{eqnarray}}
\newcommand{\EEQA}{\end{eqnarray}}
\newcommand{\define}{\stackrel{\triangle}{=}}
\bibliographystyle{IEEEtran}
\raggedbottom
\setlength{\parindent}{0pt}
\providecommand{\mbf}{\mathbf}
\providecommand{\pr}[1]{\ensuremath{\Pr\left(#1\right)}}
\providecommand{\qfunc}[1]{\ensuremath{Q\left(#1\right)}}
\providecommand{\sbrak}[1]{\ensuremath{{}\left[#1\right]}}
\providecommand{\lsbrak}[1]{\ensuremath{{}\left[#1\right.}}
\providecommand{\rsbrak}[1]{\ensuremath{{}\left.#1\right]}}
\providecommand{\brak}[1]{\ensuremath{\left(#1\right)}}
\providecommand{\lbrak}[1]{\ensuremath{\left(#1\right.}}
\providecommand{\rbrak}[1]{\ensuremath{\left.#1\right)}}
\providecommand{\cbrak}[1]{\ensuremath{\left\{#1\right\}}}
\providecommand{\lcbrak}[1]{\ensuremath{\left\{#1\right.}}
\providecommand{\rcbrak}[1]{\ensuremath{\left.#1\right\}}}
\theoremstyle{remark}
\newtheorem{rem}{Remark}
\newcommand{\sgn}{\mathop{\mathrm{sgn}}}
\providecommand{\abs}[1]{\left\vert#1\right\vert}
\providecommand{\res}[1]{\Res\displaylimits_{#1}} 
\providecommand{\norm}[1]{\left\lVert#1\right\rVert}
%\providecommand{\norm}[1]{\lVert#1\rVert}
\providecommand{\mtx}[1]{\mathbf{#1}}
\providecommand{\mean}[1]{E\left[ #1 \right]}
\providecommand{\fourier}{\overset{\mathcal{F}}{ \rightleftharpoons}}
%\providecommand{\hilbert}{\overset{\mathcal{H}}{ \rightleftharpoons}}
\providecommand{\system}{\overset{\mathcal{H}}{ \longleftrightarrow}}
	%\newcommand{\solution}[2]{\textbf{Solution:}{#1}}
\newcommand{\solution}{\noindent \textbf{Solution: }}
\newcommand{\cosec}{\,\text{cosec}\,}
\providecommand{\dec}[2]{\ensuremath{\overset{#1}{\underset{#2}{\gtrless}}}}
\newcommand{\myvec}[1]{\ensuremath{\begin{pmatrix}#1\end{pmatrix}}}
\newcommand{\mydet}[1]{\ensuremath{\begin{vmatrix}#1\end{vmatrix}}}
\numberwithin{equation}{subsection}
\makeatletter
\@addtoreset{figure}{problem}
\makeatother
\let\StandardTheFigure\thefigure
\let\vec\mathbf
\renewcommand{\thefigure}{\theproblem}
\def\putbox#1#2#3{\makebox[0in][l]{\makebox[#1][l]{}\raisebox{\baselineskip}[0in][0in]{\raisebox{#2}[0in][0in]{#3}}}}
     \def\rightbox#1{\makebox[0in][r]{#1}}
     \def\centbox#1{\makebox[0in]{#1}}
     \def\topbox#1{\raisebox{-\baselineskip}[0in][0in]{#1}}
     \def\midbox#1{\raisebox{-0.5\baselineskip}[0in][0in]{#1}}
\vspace{3cm}
\title{Assignment 1}
\author{Abhiroop Chintalapudi - AI20BTECH11005}
\maketitle
\newpage
\bigskip
\renewcommand{\thefigure}{\theenumi}
\renewcommand{\thetable}{\theenumi}
Download all python and latex codes from 
\begin{lstlisting}
https://github.com/abhiroopchintalapudi03/EE3900.git
\end{lstlisting}

\section{Problem 2.5}
Check whether\\
\begin{align}
\vec{A}=\myvec{5 \\ -2}, \vec{B}=\myvec{6 \\ 4}, \vec{C}=\myvec{7 \\ -2 }\label{eq:2.5.1}
\end{align}
are the vertices of an isosceles triangle.

\section{Solution}
Let,\\
\begin{align}
\vec{A}=\myvec{5 \\ -2}, \vec{B}=\myvec{6 \\ 4}, \vec{C}=\myvec{7 \\ -2 }\label{eq:2.5.2}
\end{align}

For the triangle to be isosceles triangle, one of\\
\begin{align}
	\norm{\vec{A} - \vec{B}} =  \norm{\vec{B} - \vec{C}} \label{eq:2.5.3}\\
	or \norm{\vec{A} - \vec{B}} =  \norm{\vec{B} - \vec{C}} \label{eq:2.5.4}\\
	or \norm{\vec{B} - \vec{C}} =  \norm{\vec{C} - \vec{A}} \label{eq:2.5.5}\\
	or \norm{\vec{C} - \vec{A}} =  \norm{\vec{A} - \vec{B}} \label{eq:2.5.6}
\end{align}

Now,\\
\begin{align}
	\vec{A-B} = \myvec{5-6 \\ (-2)-4} = \myvec{-1 \\ -6} \label{eq:2.5.7}\\
\Rightarrow \norm{A-B}^{2} = (-1)^{2}+(-6)^{2} = 37 \label{eq:2.5.8}\\
\vec{B-C} = \myvec{6-7 \\ 4-(-2)} = \myvec{-1 \\ 6} \label{eq:2.5.9}\\
\Rightarrow \norm{B-C}^{2} = (-1)^{2}+6^{2} = 37 \label{eq:2.5.10}\\
\vec{C-A} = \myvec{7-5 \\ (-2)-(-2) = \myvec{2 \\ 0}} \label{eq:2.5.11}\\
\Rightarrow \norm{C-A}^{2} = 2^{2} = 4 \label{eq:2.5.12}
\end{align}
 
 As 
\begin{align}
	 \norm{A-B}^{2} = \norm{B-C}^{2} = 37 \label{eq:2.5.13}
\end{align}
(From \eqref{eq:2.5.8} and \eqref{eq:2.5.10})\\
 $\Rightarrow$ In $\Delta ABC$ sides $AB, BC$ are equal.\\
 $\Rightarrow$ $\Delta ABC$ is an isoscles triangle.\\
 
 You can also see fom the below diagram that the triangle is an isosceles triangle with sides $AB, BC$ equal.
 
 \begin{figure}[h]
 	\centering
 	\includegraphics[width=0.7\linewidth]{"D:/SEM-3/EE3900 LINEAR SYSTEMS AND SIGNAL PROCESSING/ASSIGNMENT-1/FIGURE-1"}
 	\caption{$\Delta ABC$}
 	\label{fig:2.5}
 \end{figure}
\end{document}